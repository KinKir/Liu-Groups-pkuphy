\documentclass{beamer}
\usepackage{amsmath}
\usepackage{xeCJK}
\usepackage{graphicx}
\usefonttheme[onlymath]{serif}
\setCJKmainfont[BoldFont={STZhongsong},ItalicFont={FangSong}]{SimSun}
\setCJKsansfont[BoldFont={DengXian Bold},ItalicFont={KaiTi}]{DengXian}
\newcommand{\pdfpage}[1]{{\usebackgroundtemplate{\includegraphics[width=\paperwidth]{W4/W4_页面_#1.jpg}}\begin{frame}[plain]\label{OrigPdf#1}\end{frame}}}%将原课件的某一页作为一个全页图片插入,参数为原课件页数(图片用pdf的导出图像功能得到),注意个位数需要前面加一个零
\newcommand{\refpage}[1]{注第\ref{OrigPdf#1}页}%将原页码转化为新页码
\newcommand{\refpageN}[1]{\ref{OrigPdf#1}}%将原页码转化为新页码,只给出数字
\newcommand{\pp}[2]{\frac{\partial #1}{\partial #2}}
\newcommand{\ppat}[3]{\left.\left(\pp{#1}{#2}\right)\right|_{#3}}
\newcommand{\di}{\mathrm d}
\title{《李群和李代数》第四周课件勘误与注解}
\author{原课件:刘玉鑫老师}
\begin{document}
\maketitle
\begin{frame}{说明}
本文档基于原课件,修改了一些其中的错误并对一些不太明晰的地方进行了注解。\\
主要参考书:
\begin{itemize}
\item P. M. Cohn, \textit{Lie Groups}
\item Brain Hall, \textit{Lie Groups, Lie Algebras and Representations}
\item A. O. Barut, \textit{Theory of Group Representations and Applications}
\item 梁灿彬,《微分几何与广义相对论》
\item V. S. Varadarajan, \textit{Lie Groups, Lie Algebras and Their Representations}
\item 韩其智、孙洪洲,《群论》
\end{itemize}

本文档的页面由三种页面组成:(1)直接引用课件;(2)引用原课件,但改写了其中的错误,表现为白底黑字蓝色标题的页面;(3)对于原课件的注解,同样是白底黑字蓝色标题的页面,但标题以“注第XX页”开头。

本文中提到的页码均指本pdf页码,不是原课件页码。

部分结果来自群内讨论。如有问题,欢迎讨论。
\end{frame}
\pdfpage{01}
\pdfpage{02}
\pdfpage{03}

\begin{frame}{\refpage{03}}
	
	这一段的严格表述参见Barut书1.4节。
	
	此处第一行应该说“对给定李代数$g$及给定的一个$g$中元素$A$”,可以构造出一个这样的本征方程,解出的特征根和特征矢量显然是和$A$有关的。事实上,$\alpha$是一个关于$A$的线性函数,应该写为$\alpha(A)$更合适。
	
	这里称“$\rho$是\textbf{李代数的根}”(这个称法来自韩其智)个人认为有些不妥,因为$\rho$并非由李代数本身唯一确定。称为“李代数$g$关于$A$”的根更合适一些。
	
\end{frame}

\pdfpage{04}

\begin{frame}{\refpage{04}}
	
	此处前几行的逻辑是:对于任意一个给定的$A$,可以写出一个这样的本征方程,不同的$A$解出来的根是不一样的。可以取到一个$A$使得非简并根的数目达到最大,并且可以证明,如果$g$是半单纯的,此时非零根都是不简并的。
	
	这种基于“选一个$A$使得XXX”的逻辑似乎有一些奇怪,Barut书的表述是,先取出李代数的一个满足一定条件的子空间$h$(要求是一个极大交换子代数\footnote{“极大”指直积上任何其他维度就不交换了},还要满足一些附加条件,称为Cartan子代数\footnote{Cartan子代数是不唯一的,见\href{https://math.stackexchange.com/questions/134901/is-cartan-subalgebra-of-a-lie-algebra-unique}{\underline{\textcolor{blue}{StackExchange}}},不同的选择在一定意义下等价。}),然后定义本征方程为$[A,V]=a(A)V,\forall A\in h$。此时解出的“本征值”是$h$上的线性函数(左边是对$A$线性的,所以右边的$a(A)$应该也是),或者说是线性空间$h$的对偶空间$h^*$中的元素。假如$h$选定了一组基$H_i$,则$a(\lambda^iH_i)=\lambda^ia(H_i)$,从而$a$由其在各$H_i$上的取值完全决定,可以记$a_i=a(H_i)$,这样$a_i$这个数组就表征了$a$这个线性函数($a_i$是$a$在$H_i$的对偶基下的坐标)。$a_i$这组坐标在下面的讨论中常会用到。
	
\end{frame}

\pdfpage{05}

\begin{frame}{\refpage{05}}
	
	从单个$A$的“非简并根数量最多”能否推出这样的对易条件存疑\footnote{听说万哲先的《李代数》里有,不过我搞不到书没有查}。不过如果忽略证明的形式逻辑,我们这里需要知道的就是:
	
	\textbf{结论}:对于任意一个有限维半单纯李代数$g$,可以选取其一组基$\{H_i|i=1,\dots,l\}\cup\{E_\alpha|\alpha\in\Sigma\}$,满足:
	\begin{itemize}
		\item $[H_i,H_j]=0$,即$g$的子空间$h\equiv\langle H_i|i=1,\dots,l\rangle$是一个交换子代数(但一般不是理想)。
		\item $[A,E_\alpha]=\alpha(A)E_\alpha,\forall A\in h$,其中$\alpha\in\Sigma$是一个$h$上的线性函数,所有这样的$\alpha$构成的集合$\Sigma$称为$g$的根系,$\Sigma$的元素数等于$n-l$。\footnote{以下经常$a$,$\alpha$混记,有时指$E$的指标有时指线性函数,两种意义是一一对应的。}
	\end{itemize}
	
	然后即可开始往下推其他性质。
	
\end{frame}

\pdfpage{06}
\pdfpage{07}
\pdfpage{08}
\pdfpage{09}
\pdfpage{10}

\begin{frame}{\refpage{10}}
	此处默认了当$\alpha+\beta$是根的时候$N_{\alpha\beta}\neq 0$,事实上上面只说明了这个命题的否命题成立,无法直接说明这个命题本身成立,而下面的证明是用到了这个命题的。\textit{暂时没有在参考书里查到这个证明。}
\end{frame}

\pdfpage{11}
\pdfpage{12}

\begin{frame}{注第\refpageN{11}-\refpageN{12}页}
	
%	(以下称一个指标“$\in H$”,若其对应某个$H_i$;一个指标“$\in E$”,若其对应某个$E_\alpha$)
%	
%	第\refpageN{11}页中计算$g_{\sigma\rho}$时,对第二项的处理实际上默认了$\sigma\in E$。考虑周全情况得到的结论应该是:$g_{\sigma\rho}\neq 0$,仅当(1)$\sigma,\rho\in H$或(2)$\sigma,\rho\in E$且$\sigma=-\rho$之一成立。从而矩阵应该可以写成$g=\begin{pmatrix}g_{ij} & 0 \\ 0 & g_{\sigma\tau}\end{pmatrix}$的形式,$\det g=\det g_{ij}\det g_{\sigma\tau}$。
%	
%	(当然原证明逻辑是取定$\sigma=\alpha$然后证明这一整行为零,是没有问题的,这里只是作为补充)
	
	由于$g_{\alpha,(-\alpha)}\propto E_\alpha E_{-\alpha}$(根据度规最原始的定义显然),只要调整二者的乘积就可以使得$g_{\alpha,(-\alpha)}=1$,还剩下一个二者比例的自由度,下面的讨论中会调整它来使得$E_\alpha^\dagger=E_{-\alpha}$(\textit{这个相等似乎是没法从给定条件推出来的,只能推到两者成正比,需要通过调整二者相对大小使之相等})。
	
	这里“$a$有逆变分量”应该理解为,$a^i=g^{ij}a_j$是直接由定义(李代数空间中加上度规)得到,而恰巧(\textit{背后可能有一些深刻的原因?})它也等于对应的结构常数。
	
\end{frame}

\pdfpage{13}

\begin{frame}{\refpage{13}}
	这里用Dirac符号改记是把这个李代数中的每个元素视为了一个线性变换(算符)。这里的讨论还是在一个特殊的空间中进行的,但这一部分升降算符的结论可以适用于任意空间。
	
	这里的逻辑是:假设有一个线性空间(希尔伯特空间)$\mathcal{H}$,每一个$g$中的元素$V$都对应一个$\mathcal{H}$上的线性变换,且定义$g$中元素的相乘为其对应的线性变换的复合(李代数本身是没有定义这样一个乘法的,李乘积的性质和这里的乘法要求的性质不符),且要求$[U,V]=UV-VU$(左边为李代数中的李括号,右边为线性变换复合相减)。此时就可以继续本页中的讨论。
	
	特别地,如果令$\mathcal{H}$为李代数本身,$V$作用在$E_\alpha$上定义为$[V,E_\alpha]$,那么可以验证这确实满足上面要求的条件。这时的特别之处在于,由于$[H_i,E_\alpha]=a_iE_\alpha$,(作为$\mathcal{H}$中元素的)$E_\alpha$恰好是(作为李代数$g$中元素、对应于$\mathcal{H}$上线性变换的)$H_i$的共同本征态,本征值排列起来恰好就是根矢量,从而非零根和$H_i$的共同非零本征态建立了一一对应关系;而各$H_j$(作为Hilbert空间中的量子态)则是各$H_i$算符的$l$重简并零本征态。下面证明过程中有些地方用到了这个结论。
	
\end{frame}

\pdfpage{14}

\begin{frame}{\refpage{14}:关于$E_\alpha=E_{-\alpha}^\dagger$的进一步说明}
	
	由于$H_i$的本征值为实数的本征态构成整个希尔伯特空间的一组基,可以直接由定义验证各个$H_i$确实是这样希尔伯特空间中的厄米算符。

	取任意非零根$\alpha$和另一非零根$\mu$。我们知道有$E_\alpha|\mu\rangle=N_{\alpha,\mu}|\mu+\alpha\rangle$,$E_{-\alpha}|\mu+\alpha\rangle=N_{-\alpha,\mu+\alpha}|\mu\rangle$。
	从而有
	
	\begin{equation*}
	\begin{cases}
	\langle \nu | E_\alpha \mu \rangle=N_{\alpha,\mu}\delta_{\nu,\mu+\alpha}\\
	\langle E_{-\alpha} \nu | \mu \rangle=N_{-\alpha,\mu+\alpha}^*\delta_{\nu,\mu+\alpha}
	\end{cases}
	\end{equation*}
	如果令$E_\alpha'=\sqrt{\frac{N_{-\alpha,\mu+\alpha}^*}{N_{\alpha,\mu}}}E_\alpha$,$E_{-\alpha}'=\sqrt{\frac{N_{\alpha,\mu}}{N_{-\alpha,\mu+\alpha}^*}}E_{-\alpha}$,则其他关系式均得到保持,且有
	\begin{equation*}
	\langle \nu | E_{\alpha} \mu \rangle=\langle E_{-\alpha} \nu | \mu \rangle\forall \nu
	\end{equation*}
	\textit{好像上面这么做还需要假设$\frac{N_{-\alpha,\mu+\alpha}^*}{N_{\alpha,\mu}}$是实数?}
	
\end{frame}

\begin{frame}{\refpage{14}:关于$E_\alpha=E_{-\alpha}^\dagger$的进一步说明}
	
	\textit{然后……似乎无法说明此时对于其他的内积这个厄米条件也满足orz待补充……}
	
	%如果另有一非零根$\rho$,使得$\rho+\mu=\lambda$也是根,则可以写$|\lambda\rangle=\frac{1}{N_{\rho,\mu}}E_{\rho}|\mu\rangle$(简单起见$\frac{1}{N_{\rho,\mu}}=:n$),于是
	%\begin{equation*}
	%\langle \nu | E_{\alpha} \lambda \rangle=n\langle \nu |E_\alpha E_{\rho} \mu \rangle
	%\end{equation*}
\end{frame}

\pdfpage{15}
\pdfpage{16}
\pdfpage{17}
\pdfpage{18}

\begin{frame}{\refpage{18}}
	这里的证明和韩其智书的证明其实是一样的,只要将本征态和非零根进行对应,则韩书中的$\mu_j$就是这里的$|N_{\alpha,\mu+(p-j)\alpha}|^2$。
	
	这个定理的结论其实不只是$\frac{2(a,b)}{(a,a)}$为整数,更重要地,它等于$q-p$。于是,如果已知$q,p$,则可以反推这个值;如果已知这个值,也可以得到$q\geq \frac{2(a,b)}{(a,a)}$,$p\geq -\frac{2(a,b)}{(a,a)}$。特别地,当$q=p$时,必有$(a,b)=0$。

	关于$(a,a)\neq 0$的证明:如果某个根$a$满足$(a,a)=0$,由上推导可以知道再取任意其他根$b$都会有$(a,b)=0$。这表明\footnote{这里明显跳步了,但是中间的步骤我没想到怎么补上,就当它成立吧……}$g_{ij}a^j=0$,也就是$\det g_{ij}=0$。结合下面关于度规张量性质的分析,这进一步表明$\det g=0$,与半单纯性矛盾。
	
%	\textit{这里似乎没有直接推出$(a,a)\neq 0$(至少我没有看懂……),仅从$g$非退化是不能推出$(a,a)\neq 0$的,反例:$g=\begin{pmatrix}1 & 0 \\ 0 & -1\end{pmatrix},a=\begin{pmatrix}1 \\ 1\end{pmatrix}$,需要加上$g$正定的条件才可以,但是这里似乎无法保证$g$正定?}
%	
%	\textit{利用定义和前面的关系式可以改写$(a,a)E_\alpha=[E_\alpha,[E_\alpha,E_{-\alpha}]]=E_\alpha^2|-\alpha\rangle$,如果利用“只要作用后的下标是根则作用结果必不为零”,则可以推出$(a,a)\neq 0$,不过前者也还没有证明。}
\end{frame}

\pdfpage{19}
\pdfpage{20}
\pdfpage{21}

\begin{frame}{\refpage{21}}
	这里推出“根系必定包含$2a$”,其实是在之前构造根系的公式里,令$b=ka$,则$q-p=\frac{2(a,b)}{(a,a)}=2k$,于是必有$q\geq 2k,p\geq -2k$,从而可以推出这个根系必定延伸到包含$2a$。
\end{frame}

\pdfpage{22}
\pdfpage{23}

\begin{frame}{\refpage{23}}
	\textit{这里前一半证明的$|q-p|\leq 3$似乎不需要,因为它完全可以由后面的结论$q+p\leq 3$推出。而且这里似乎也没有证出$m,m'$的普遍情况?$(b,b)=4(a,a)$并不能推出$b=2a$?}
\end{frame}

\pdfpage{24}

\begin{frame}{\refpage{24}}
	一般地,如果一个根系包含至少五个根,则必然存在一个整数$n$,使得$b+(n-2)a$到$b+(n+2)a$都在这个根系内,取$\frac{2(b+(n-2)a,b+na)}{(b+na,b+na)}$和$\frac{2(b+(n+2)a,b+na)}{(b+na,b+na)}$,结论一致。
\end{frame}

\pdfpage{25}
\pdfpage{26}
\pdfpage{27}
\pdfpage{28}
\pdfpage{29}
\pdfpage{30}
\pdfpage{31}
\pdfpage{32}


\end{document}