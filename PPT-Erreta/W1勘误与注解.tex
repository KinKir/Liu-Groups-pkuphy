\documentclass{beamer}
\usepackage{amsmath}
\usepackage{xeCJK}
\usepackage{graphicx}
\usefonttheme[onlymath]{serif}
\setCJKmainfont[BoldFont={STZhongsong},ItalicFont={FangSong}]{SimSun}
\setCJKsansfont[BoldFont={DengXian Bold},ItalicFont={KaiTi}]{DengXian}
\newcommand{\pdfpage}[1]{{\usebackgroundtemplate{\includegraphics[width=\paperwidth]{W1/W1_页面_#1.jpg}}\begin{frame}[plain]\label{OrigPdf#1}\end{frame}}}%将原课件的某一页作为一个全页图片插入,参数为原课件页数(图片用pdf的导出图像功能得到),注意个位数需要前面加一个零
\newcommand{\refpage}[1]{第\ref{OrigPdf#1}页}%将原页码转化为新页码
\newcommand{\refpageN}[1]{\ref{OrigPdf#1}}%将原页码转化为新页码,只给出数字
\newcommand{\pp}[2]{\frac{\partial #1}{\partial #2}}
\newcommand{\ppat}[3]{\left.\left(\pp{#1}{#2}\right)\right|_{#3}}
\newcommand{\di}{\mathrm d}
\title{《李群和李代数》第一周课件勘误与注解}
\author{原课件:刘玉鑫老师}
\begin{document}
\maketitle
\begin{frame}{说明}
本文档基于原课件,修改了一些其中的错误并对一些不太明晰的地方进行了注解。\\
主要参考书:
\begin{itemize}
\item P. M. Cohn, \textit{Lie Groups}
\item Brain Hall, \textit{Lie Groups, Lie Algebras and Representations}
\item V. S. Varadarajan, \textit{Lie Groups, Lie Algebras and Their Representations}
\item 韩其智、孙洪洲,《群论》
\end{itemize}

本文档的页面由三种页面组成:(1)直接引用课件;(2)引用原课件,但改写了其中的错误,表现为白底黑字蓝色标题的页面;(3)对于原课件的注解,同样是白底黑字蓝色标题的页面,但标题以“注第XX页”开头。

本文中提到的页码均指本pdf页码,不是原课件页码。

部分结果来自群内讨论,主要为陆仲豪和唐子骞同学。如有问题,欢迎讨论。
\end{frame}
\begin{frame}{引言部分}
	引言部分讲的粒子物理相关内容我也不太懂orz欢迎有大佬前来注解……
\end{frame}
\pdfpage{02}
\pdfpage{03}
\pdfpage{04}
\pdfpage{05}
\pdfpage{06}
\pdfpage{07}
\pdfpage{08}
\pdfpage{09}
\pdfpage{10}
\begin{frame}{注\refpage{10}:关于条件(3)(4)可换的证明}
	参考:Joseph J. Rotman, \textit{An Introduction to the Theory of Groups}, Theorem 1.12
	
	\textbf{欲证明}:如果一个带乘法的集合$G$满足乘法结合律、存在左单位元和存在左逆元,则这个左单位元同时也是右单位元、左逆元同时也是右逆元,且唯一。\\
	\textbf{严格表述}:如果有一个集合$G$和其上的二元运算(称之为乘法)$\cdot:G\times G\to G$,满足$(ab)c=a(bc),\forall a,b,c\in G$,并且存在$e\in G$,使得:$ea=a,\forall a\in G$,且$\forall a\in G, \exists b\in G, \text{s.t.} ba=e$,则,$ae=ea=a,\forall a\in G$,且对每一个$a\in G$,存在唯一的$a^{-1}\in G$,使得$aa^{-1}=a^{-1}a=e$。
	
	\textbf{证明:}
	
	引理:如果$x\in G$满足$xx=x$,则$x=e$。\\
	证明:取一个$y$(这里不假设唯一)使得$yx=e$,则$x=ex=(yx)x=y(xx)=yx=e$,引理证毕。
	
\end{frame}
	
\begin{frame}{注\refpage{10}:关于条件(3)(4)可换的证明(续)}
	
	若$ba=e$,则$ab=a(eb)=a(ba)b=(ab)(ab)$,由引理知$ab=e$,从而左逆元同时也是右逆元。
	
	对于$a\in G$,取$b$为$a$的一个逆元,则$ae=a(ba)=(ab)a=a$,从而$e$同时也是右单位元。
	
	如果存在$b,b'$满足$ab=ab'=e$,则$b=be=b(ab')=(ba)b'=eb'=b'$,于是逆元唯一。\qedsymbol
	
\end{frame}

\begin{frame}{群的概念(续)}
上述条件(3)(4)还可以改写为:$\forall a,b\in G$,方程$ax=b$和方程$ya=b$分别在$G$中有且只有一个解。

\textbf{证明:}由原(3)(4)推出这一条件显然(乘逆元过去即可);如果这一条件成立,任取$a\in G$,由条件知有且只有一个$e_a\in G$使得$e_aa=a$。再任取一个$b\in G$,取$y$为$ay=b$的解,则$e_ab=e_a(ay)=(e_aa)y=ay=b$,于是$e_a$是整个集合上的左单位元。又由条件知每个元素都有左逆元,于是推得(3)(4)。\qedsymbol

用上述条件来判断是否成群的例子:设
\begin{equation*}
a=\begin{pmatrix}1 & 1 \\ 0 & 0\end{pmatrix}, b=\begin{pmatrix}0 & 0 \\ 1 & 1\end{pmatrix}
\end{equation*}
则有$aa=a,ab=a,bb=b,ba=b$,满足乘法的封闭性。但是,方程$ax=a$和$bx=b$有两个解、方程$ax=b$和$bx=a$无解,从而不满足成群条件,$\{a,b\}$不成群。

又如:有二维平面中的转动矩阵$R_z(\theta)=\begin{pmatrix}\cos\theta & -\sin\theta \\ \sin\theta & \cos\theta\end{pmatrix}$,则所有这样的转动$\{R_z(\theta)|\theta\in\mathbb{R}\}$成群($SO(2)$群)。
\end{frame}

\begin{frame}{群的概念(续)}
观察:不成群的$a,b$,有$\det a=\det b=0$;成群的$R_z(\theta)$,有$\det R_z(\theta)=1\neq 0$。事实上,有一般的结论:所有$n$阶的非奇异矩阵的集合构成一个群($GL(n)$群);但并不是所有的矩阵群都一定由非奇异矩阵构成。反例:$\left\{\begin{pmatrix}1 & 0 \\ 0 & 0\end{pmatrix}\right\}$。

一般讨论:\textit{猜测,总可以取到某个子空间使得在这个子空间上$G$中每个元素都是可逆的?待验证。}
\end{frame}

\pdfpage{12}

\begin{frame}{注\refpage{12}}
	表述“$GL(n,\mathbb C)$有$2n^2$个\textbf{群元}”有些引起混淆,“群元”应该指群的元素,这里应该指“矩阵元”,下同。
\end{frame}

\pdfpage{13}
\pdfpage{14}

\begin{frame}{注\refpage{14}}
	我们有$(UU^\dagger)_{ij}=U^*_{ik}U_{jk}=\delta_{ij}$,共$n^2$个方程;其中$i,j$互换给出$0=U^*_{jk}U_{ik}=(U^*_{ik}U_{jk})^*$,故和互换前的方程不独立,从而非对角元共$\frac{n^2-n}{2}$个方程,每个方程均有实部和虚部;对角元方程$U^*_{ik}U_{ik}=1$(不对$i$求和),左边必定为实,从而只给出实部的贡献。
	
	事实上,这里也只验证了“有这么多方程是等价的”,还没有验证剩下的这些都是相互独立的,验证起来可能比较困难(我没有试),更简便的做法应该是计算这个矩阵群的李代数的维度,然后利用李代数维度等于李群维度。下同。
\end{frame}

\pdfpage{15}
\pdfpage{16}

\begin{frame}{注第\refpageN{15}-\refpageN{16}页}
	$U(p,q)$不是“特殊地”,是在另一种内积定义下的保内积线性变换群。
	
	第\refpageN{16}页第一行应该为$(X,Y)=g^{ij}x_i^*y_j=X^\dagger GY$,这里$X$、$Y$都表示列矢量。
\end{frame}

\pdfpage{17}

\begin{frame}{注\refpage{17}}
	这里讨论独立变元数量与酉群情况类似,注意此时对角元不再有为实部的保证。
	
	记号有一点混淆,一般来说$SO(n)$和$O(n)$指实数的,这里定义的$O(n)$记作$O(n,\mathbb C)$。
\end{frame}

\pdfpage{18}
\pdfpage{19}

\begin{frame}{注\refpage{19}}
	
	这里写的是$SP(2n,\mathbb C)$(有的文献记为$SP(n,\mathbb C)$),注意在这个情况下内积定义仍然是直接相乘,不取复共轭!(这里没有写错)
	
	倒数第四行应该没有那个转置,已经写成指标求和形式就没有$x$和$M$谁先谁后的问题,乘法顺序由$x$和$M$的哪个指标缩并来体现。
	
	写到倒数第三行时,等号左边的求和还是对$j,k=1\sim 2n$的,所以说直接得到的等式应该是
	\begin{equation*}
	\sum_{i=1}^{n} M_{ij}M_{(n+i)k}-M_{ik}M_{(n+i)j}=\delta_{(n+j)k}-\delta_{(n+k)j}
	\end{equation*}
	(事实上就是$M^TG_{SP}M=G_{SP}$)
	
	这些方程中,$j=k$的情况两边都恒等于零,$j,k$互换两边都反号,所以总共$\frac{2n(2n-1)}{2}$个独立的式子,每个均有实部和虚部。
	
	这些线性变换又称哈密顿变换(对应哈密顿力学中的正则变换),带这样内积的空间称为辛空间。可参考高代教材和Goldstein经典力学。
	
\end{frame}

\pdfpage{20}
\pdfpage{21}

\begin{frame}{注\refpage{21}}
	紧致性的数学定义:一个拓扑空间称为紧致的,若其任意的开覆盖有有限子覆盖。(关于拓扑学的基本知识快速入门可以参考Nakahara, \textit{Geometry, Topology and Physics},如果需要进一步的知识可以参看尤承业《基础拓扑学讲义》、Munkres, \textit{Topology})
	
	特别地,\textbf{一个$\mathbb R^{n}$的子集是紧致的,当且仅当它是有界闭集}。
	
	其中,$A\subset \mathbb R^n$为闭集,若对于任意的$x\notin A$,存在$\epsilon>0$,使得以$A$中任意点和$x$的距离不小于$\epsilon$。(如,$r<1$的球,球面上的点就不满足这个条件,所以它不是闭集;$r\leq 1$的球则满足这个条件,是闭集。)
	
	集合$A\subset \mathbb R^n$有界指,存在$M$,使得
	\begin{equation*}
	\forall x=(x_1,x_2,\dots,x_n)\in A, \sqrt{x_1^2+x_2^2+\dots+x_n^2}<M
	\end{equation*}
	
	把矩阵的各个矩阵元排起来成为数组,则可以视为$\mathbb R^{n}$的子空间(如果有虚数实部虚部拆开),而有界等价于各个矩阵元分别有界。行列式条件和度规条件限制给出的集合一般都是闭集,可以理解为,对矩阵元的若干方程约束相当于给出“超曲面”,而诸如三维空间中的曲面、曲线都是闭集。
\end{frame}

\pdfpage{22}
\pdfpage{23}
\pdfpage{24}
\pdfpage{25}

\begin{frame}{注\refpage{25}:前两行结论的推导}
	前两行,首先由于,假设$\varphi(\alpha,\beta)$是光滑的(这个应该是假设而非推论),则在$\alpha,\beta=0$附近有:
	\begin{equation*}
	\varphi(\alpha,\beta)=\ppat{\varphi(x,y)}{x}{(x,y)=(0,0)}\alpha+\ppat{\varphi(x,y)}{y}{(x,y)=(0,0)}\beta
	\end{equation*}
	又由于
	\begin{equation*}
	\varphi(\alpha,0)=\varphi(0,\alpha)=\alpha
	\end{equation*}
	两边在零处求导得
	\begin{equation*}
	\ppat{\varphi(x,y)}{x}{(x,y)=(0,0)}=\ppat{\varphi(x,y)}{y}{(x,y)=(0,0)}=1
	\end{equation*}
	从而在原点附近有$\varphi(\alpha,\beta)=\alpha+\beta$。由此亦知$-\alpha$是$\alpha$的一个逆元,由逆元唯一性得到$\overline{\alpha}=-\alpha$。
	
	这里的式子里变量其实都是$r$元向量,但经过简单改写知结果没有区别。
	
\end{frame}

\begin{frame}{注\refpage{25}:李群的定义}
	李群的严格定义为,一个集合上同时有群和微分流形的结构,且群乘法作为微分流形上的运算是解析的。当然我们不会用这个定义。
	
	按照合成函数来定义李群的话,应该要求合成函数具有如下几个性质:(此处参考Cohn第2.7节)
	\begin{enumerate}
		\item $\varphi(\alpha,\beta)$是关于自变量的光滑(任意阶可微)函数
		\item $\varphi(\alpha,\varphi(\beta,\gamma))=\varphi(\varphi(\alpha,\beta),\gamma)$
		\item $\varphi(\alpha,0)=\alpha$
		\item $\forall \alpha,\exists\beta, \text{s.t.}\varphi(\alpha,\beta)=0$(\textit{如果不是讨论局部李群是否还需要雅可比行列式条件?处处有逆能否推出雅可比行列式非零?存疑})
	\end{enumerate}
	后三个条件也就是成群条件。
\end{frame}

\begin{frame}{注\refpage{25}:李群的定义与局部李群}
	物理上一般用坐标来刻画李群,比如三维转动$SO(3)$群一般用欧拉角来刻画。但是一般来说,李群是无法与$\mathbb R^n$(或其某个子集)建立连续的双射的(背后原因是两者拓扑结构不同),比如欧拉角,其映射确实连续,但它不是一一的——在进动角为零处自转角和章动角不独立,这会导致这点附近出现一些不好的现象,比如用坐标表述的逆元不唯一。
	
	虽然一般来说无法定义全局坐标,但根据流形的定义(其中包括了一条保证其局部性质与$\mathbb R^n$相同),总是可以在每一个点都选取一个邻域,并在上面定义坐标,此时坐标和李群元素的对应是一一的,于是可以采取上面这一套描述体系。这样得到的系统称为\textbf{局部李群}。
\end{frame}

\begin{frame}{注\refpage{25}:李群的定义与局部李群}
	局部李群的定义为:存在一个在原点处解析的合成函数$\varphi(\alpha,\beta)$,在原点的一个邻域内满足:
	\begin{enumerate}
		\item $\varphi(\alpha,\varphi(\beta,\gamma))=\varphi(\varphi(\alpha,\beta),\gamma)$
		\item $\varphi(\alpha,0)=\alpha$
		\item $\det\ppat{\varphi^i(\alpha,\beta)}{\beta^j}{\alpha=\beta=0}\neq 0$
	\end{enumerate}
	其中最后一条结合隐函数定理保证了原点附近逆元的存在性。
	
	可以证明,一个局部李群可以唯一地延拓为一个连通(在讨论$\mathbb R^n$的子集时,连通等价于集合中任意两点可以用一条连续的路径连接起来;后者在一般的拓扑学语境下称为道路连通)的全局李群(Cohn Theorem 2.7.1),从而一般来说在局域坐标下讨论得到的结果是普适的。当需要讨论两个有限远处的点之间的映射等时,可以在两点处分别建立局域坐标讨论,结果一般和当作只有一套局域坐标等价。
\end{frame}

\begin{frame}{注\refpage{25}:李群与李变换群}
	最后几行写得有一点乱,应该是现有李群的概念后有李变换群的概念。也可以说李群是抽象的,李变换群是特殊情况。
	
	假设给定李群$G$和另一个集合$V$(可以是线性空间,也可以不是),并且每一个$A(\alpha)\in G$对应一个$V$上的变换(即$V$到自身的映射),且有结合律:$[A(\alpha)B(\alpha)]v=A(\alpha)[B(\alpha)v]$(有的文献似乎反过来定义),则称$G$是李变换群。(c.f. 群在集合上的作用)(当$V$有附加结构时可能会对这个作用提出更多要求)
	
	例子:$G=SO(3),V=\mathbb R^3$,群元的作用即为矢量作对应的旋转。
\end{frame}

\begin{frame}{1.3 无穷小生成元}
	{\large 一、李变换群的无穷小生成元}
	假设有李群$G$,其作用在集合$V$上,假设$V$也是一个微分流形,即上面有坐标并且可以对上面的函数求导。对于$Y\in V$和$B(\beta)\in G$,记$B(\beta)Y=:f(\beta,Y)$。取$X_0\in V,A(\alpha)\in G$,并令$X=A(\alpha)X_0$。现在考虑对$\alpha$进行一微小变动,这一微小变动有两种可能的表述形式,第一种是:新的$X$仍然是从$X_0$经过一次变动得到,但是变换参数有一微小变动,于是$X+\di X=f(\alpha+\di\alpha,X_0)$;第二种是,新的$X$是由$X_0$连续两次作用、即旧的$X$上再加上一个无穷小群作用得到的。此时$X+\di X=f(\delta\alpha,X)$。两个无穷小参数$\di\alpha\neq\delta\alpha$。
	
	我们有:
	\begin{equation*}
	X^i+\di X^i=f^i(\delta\alpha, X)=f^i(0,X)+\ppat{f^i(\beta,X)}{\beta^\sigma}{\beta=0}\delta\alpha^\sigma
	\end{equation*}
	令$U^i_\sigma(X)=\ppat{f^i(\beta,X)}{\beta^\sigma}{\beta=0}$,则有$\di X^i=U^i_\sigma(X)\delta\alpha^\sigma$。	
	
	其中$\sigma$是$G$的坐标对应的指标,$i$是$V$的坐标对应的指标。
\end{frame}

\begin{frame}{1.3 无穷小生成元}
	{\large 一、李变换群的无穷小生成元}
	而同时又有
	\begin{equation*}
	X+\di X=A(\alpha+\di\alpha)X=A(\delta\alpha)[A(\alpha) X]
	\end{equation*}
	(这里的“乘法”都代表“作用”)
	
	即$\alpha+\di\alpha=\varphi(\delta\alpha,\alpha)$,作小量展开得到
	\begin{equation*}
	\di\alpha^\mu=V^\mu_\sigma(\alpha)\delta\alpha^\sigma
	\end{equation*}
	其中$V^\mu_\sigma(\alpha)=\ppat{\varphi^\mu(\beta,\alpha)}{\beta^\sigma}{\beta=0}$。
	
	假设$V$作为矩阵可逆,记其逆矩阵为$\Lambda$,则有$\delta\alpha^\sigma=\Lambda^\sigma_\mu(\alpha)\di\alpha^\mu$。
\end{frame}

\pdfpage{28}
\pdfpage{29}
\pdfpage{30}
\pdfpage{31}
\pdfpage{32}
\pdfpage{33}
\pdfpage{34}
\pdfpage{35}
\pdfpage{36}


\end{document}